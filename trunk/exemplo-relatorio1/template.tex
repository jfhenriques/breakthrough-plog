\documentclass[15pt,a4paper]{article}

%use the english line for english reports
%usepackage[english]{babel}
\usepackage[portuguese]{babel}
\usepackage[utf8]{inputenc}
\usepackage{indentfirst}
\usepackage{graphicx}
\usepackage{verbatim}


\begin{document}

\setlength{\textwidth}{16cm}
\setlength{\textheight}{22cm}

\title{\Huge\textbf{Título do Trabalho}\linebreak\linebreak\linebreak
\Large\textbf{Relatório Final}\linebreak\linebreak
\includegraphics[height=6cm, width=7cm]{feup.pdf}\linebreak \linebreak
\Large{Mestrado Integrado em Engenharia Informática e Computação} \linebreak \linebreak
\Large{Programação em Lógica}\linebreak
}

\author{\textbf{Grupo xx:}\\ Nome 1 - Número 1 \\ Nome 2 - Número 2 \\\linebreak\linebreak \\
 \\ Faculdade de Engenharia da Universidade do Porto \\ Rua Roberto Frias, s\/n, 4200-465 Porto, Portugal \linebreak\linebreak\linebreak
\linebreak\linebreak\vspace{1cm}}
%\date{Junho de 2007}
\maketitle
\thispagestyle{empty}

%************************************************************************************************
%************************************************************************************************

\newpage

\section*{Resumo}
Descrever muito sumariamente o trabalho. Deve ser suficiente para o leitor decidir se lê ou não o resto do relatório.

Não deve incluir nenhuma referência ao facto de ser feito no âmbito de uma cadeira.

\newpage

\tableofcontents

%************************************************************************************************
%************************************************************************************************

%*************************************************************************************************
%************************************************************************************************

\newpage

\section{Introdução}
Descrever os objectivos e motivação do trabalho, e estrutura do mesmo (Último parágrafo deve indicar a estrutura do relatório). Não deve incluir nenhuma referência ao facto de ser feito no âmbito de uma cadeira.

Todas as referências da bibliografia devem ser referidas no relatório \cite{CodigoSite, CodigoLivro}. Devem ser incluídas referências bibliográficas correctas e completas (consultar os docentes em caso de dúvida). Páginas da wikipedia não são consideradas referências válidas.

Todas as figuras devem ser referidas no texto. 

\section{Descrição do Problema}
Descrever sucintamente o jogo, a sua  história e,  principalmente, as suas regras (pode ser idêntico ao texto do relatório intercalar).

\section{Arqutectura do Sistema}
Descrever em linhas gerais o sistema e os módulos que o constituem. Deve ser abordada a comunicação com o visualizador, que mesmo que ainda não esteja implementada, já deverá estar pensada. Assim, deve ser incluída a sintaxe das mensagens a trocar com o visualizador.

\section{Módulo de Lógica do Jogo}
Descrever o projecto e implementação do módulo Prolog, incluindo a forma de representação do estado do tabuleiro,  verificação do cumprimento das regras do jogo, determinação do final do jogo e cálculo das jogadas a realizar pelo computador  utilizando diversos níveis de jogo.

\subsection{Representação do Estado do Jogo} \textit{estado(?Tabuleiro).}
\subsection{Visualização do Estado do Jogo} \textit{visualiza\_estado(+Tabuleiro).}
\subsection{Validação de Jogadas} \textit{movimento\_valido(?Jogada, +Tabuleiro).}
\subsection{Execução de Jogadas}\textit{executa\_movimento(+Jogada, + Tabuleiro, -NovoTabuleiro).}
\subsection{Lista de Jogadas Válidas}\textit{lista\_jogadas(+Tabuleiro, -ListaJogadas).}
\subsection{Avaliação do Tabuleiro}\textit{avalia\_tabuleiro(+Tabuleiro, -Valor).}
\subsection{Final do Jogo} \textit{fim\_jogo(+Tabuleiro, -Vencedor).}
\subsection{Cálculo da Jogada do Computador}\textit{calcula\_jogada(+Nível, +Tabuleiro, -Jogada).}
\subsection{Recepção de mensagem do visualizador}\textit{recebe\_mensagem(+ Mensagem, - Resposta).}

\section{Interface com o Utilizador}
Descrever  resumidamente o módulo de interface com o utilizador (visualizador) desenvolvido (caso exista).

\section{Conclusões e Perspectivas de Desenvolvimento}
Que conclusões  retira deste projecto? Como poderia melhorar o trabalho desenvolvido?

\clearpage
\addcontentsline{toc}{section}{Bibliografia}
\renewcommand\refname{Bibliografia}
\bibliographystyle{plain}
\bibliography{myrefs}

\newpage
\appendix
\section{Nome do Anexo A}
Código Prolog implementado devidamente comentado e outros elementos úteis que não sejam essenciais ao relatório.

\end{document}
